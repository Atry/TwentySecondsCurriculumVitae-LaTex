%%%%%%%%%%%%%%%%%%%%%%%%%%%%%%%%%%%%%%%%%
% Twenty Seconds Resume/CV
% LaTeX Template
% Version 1.1 (8/1/17)
%
% This template has been downloaded from:
% http://www.LaTeXTemplates.com
%
% Original author:
% Carmine Spagnuolo (cspagnuolo@unisa.it) with major modifications by 
% Vel (vel@LaTeXTemplates.com)
%
% License:
% The MIT License (see included LICENSE file)
%
%%%%%%%%%%%%%%%%%%%%%%%%%%%%%%%%%%%%%%%%%

%----------------------------------------------------------------------------------------
%	PACKAGES AND OTHER DOCUMENT CONFIGURATIONS
%----------------------------------------------------------------------------------------

\documentclass[letterpaper]{twentysecondcv} % a4paper for A4


\usepackage{CJKutf8}
\usepackage[utf8]{inputenc} % optional

\usepackage{listings}
\lstset{
  language=Scala,
  basicstyle=\ttfamily,
  breaklines=true,
  columns=fixed,
  basewidth=0.5em,
  captionpos=b,
  tabsize=2,
  frame=top,frame=bottom,
}

%----------------------------------------------------------------------------------------
%	 PERSONAL INFORMATION
%----------------------------------------------------------------------------------------

% If you don't need one or more of the below, just remove the content leaving the command, e.g. \cvnumberphone{}

\profilepic{575x575.png} % Profile picture

\cvname{杨博 (Yang Bo)} % Your name
\cvjobtitle{Lead Consultant} % Job title/career

\cvdate{22 February 1986} % Date of birth
\cvaddress{} % Short address/location, use \newline if more than 1 line is required
\cvnumberphone{+86 18126548504} % Phone number
\cvsite{http://hen.la} % Personal website
\cvmail{pop.atry@gmail.com} % Email address

%----------------------------------------------------------------------------------------
\begin{document}
\begin{CJK}{UTF8}{gbsn}

%----------------------------------------------------------------------------------------
%	 ABOUT ME
%----------------------------------------------------------------------------------------

\aboutme{Yang Bo is a technical lead who designed and implemented many commercial and open-source products in
various domains, including mobile and web development, online game and machine learning. He now focuses on applying
meta-programming and functional programming paradigms to distributed machine learning system.} % To have no About Me section, just remove all the text and leave \aboutme{}

%----------------------------------------------------------------------------------------
%	 SKILLS
%----------------------------------------------------------------------------------------

% Skill bar section, each skill must have a value between 0 an 6 (float)
\skills{{Agile Project Management/4.3},{Software Design/5.8},{Machine Learning/3},{Scala/5.5}}

%------------------------------------------------

% Skill text section, each skill must have a value between 0 an 6
\skillstext{{Scala.js/5.5},{Haxe/5.5},{OpenCL/5},{C++/4},{Java/4},{JavaScript/4},{DevOps/4},{Play Framework/3},{ActionScript/4},{Testing/3},{C/3},{Spark/3},{Akka/3}}

%----------------------------------------------------------------------------------------

\makeprofile % Print the sidebar





\section{Open-source projects}

\begin{twenty}
	\twentyitem{2016-present}{\href{http://deeplearning.thoughtworks.school/}{DeepLearning.scala}}{Scala}{is a simple library for creating complex neural networks.}
	\twentyitem{2015-present}{\href{https://github.com/ThoughtWorksInc/Binding.scala}{Binding.scala}}{Scala}{is a popular Reactive Web Framework written in Scala.js.}
	\twentyitem{2015-present}{\href{https://github.com/ThoughtWorksInc/Each}{ThoughtWorks Each}}{Scala}{provides the \lstinline{each} notation for creating monadic expressions in Scala.}
	\twentyitem{2015-2016}{\href{https://thoughtworksinc.github.io/microbuilder/1-overview.html}{Microbuilder}}{Haxe / Scala / Java / JavaScript}{is a toolkit that helps you build system across micro-services implemented in various languages communicating via RESTful JSON API.}
	\twentyitem{2014}{\href{https://github.com/qifun/stateless-future}{Stateless Future}}{Scala}{is a set of Domain-specific language for asynchronous programming, in the pure functional flavor.}
	\twentyitem{2012-present}{\href{https://github.com/Atry/haxe-continuation}{haxe-continuation}}{Haxe}{provides \lstinline{async}/\lstinline{await} syntax for Haxe.}
	\twentyitem{2010-present}{\href{https://github.com/Atry/protoc-gen-as3/}{protoc-gen-as3}}{ActionScript3}{is a Protocol Buffers plugin for ActionScript 3.}
\end{twenty}

\section{Publications}

\begin{twentyshort} % Environment for a short list with no descriptions
	\twentyitemshort{2017}{\href{http://deeplearning.thoughtworks.school/assets/paper.pdf}{DeepLearning.scala 2.0: Statically Typed Neural Networks}}
	\twentyitemshort{2017}{\href{http://www.infoq.com/articles/more-than-react-part-i}{More than React}}
	\twentyitemshort{2009}{\href{https://books.google.com/books/about/Flex_3\%E6\%9D\%83\%E5\%A8\%81\%E6\%8C\%87\%E5\%8D\%97.html?id=dNUsQwAACAAJ}{Adobe Flex 3: Training from the Source} (translator)}
\end{twentyshort}

\section{Talks}

\begin{twenty}
	\twentyitem{2017}{\href{http://bdtc2017.bigdataforum.org.cn/m/zone/bdtc2017/guest_detail?mid=1547&id=5932}{Deeplearning.Scala——开源深度学习框架思考与实践}}{DBTC}{}
	\twentyitem{2017}{\href{https://thestrangeloop.com/2017/monadic-deep-learning.html}{Monadic Deep Learning}}{Strange Loop}{}
	\twentyitem{2017}{\href{http://myslide.cn/slides/6644}{神经网络与函数式编程}}{ArchData Summit · Beijing}{}
	\twentyitem{2016}{\href{http://www.infoq.com/cn/presentations/more-than-async}{More than Async}}{QCon · Beijing}{}
	\twentyitem{2015}{\href{http://www.ecug.org/2015:home}{Specific-domain extension in an universal language}}{ECUG Con}{}
\end{twenty}


\section{Experience}

\begin{twenty} % Environment for a list with descriptions
	\twentyitem{2015-present}{Thoughtworks Inc}{Lead Consultant}{
		I maintained some open-source projects sponsored by ThoughtWorks.
		I also provided consulting and delivery service for various clients in different domains including mobile, web and big data.
	}
	\twentyitem{2014}{Shenzhen QiFun Network Corp., LTD}{Chief Programmer}{I led the Q-Force Team in developing game engine along with its related tools, and two mobile games.}
	\twentyitem{2011-2013}{Shenzhen Putaoteng Network Technology
	Co., Ltd.}{Co-founder}{I led the start-up team in developing a 3D side-scrolling game engine VinyHome and a social game 男搭女配.}
	\twentyitem{2008-2011}{NetEase, Inc.}{Chief Programmer}{I participated in various product teams as a developer or the technical lead, including 战国风云 (web game), 卡牌对决 (web game), Deepcold (3D game engine).}
	\twentyitem{2007}{Beijing HiPiHi Information Technology
	Corp.,Ltd}{Software Engineer}{I developed variant components of both client-side and server-side for a 3D virtual world, HiPiHi World.}
	\twentyitem{2006-2007}{Beijing AutoNavi Software Co., Ltd.}{Software Engineer}{As a developer, I developed some components for a 3D navigation software.}
	
\end{twenty}

\section{Education}

\begin{twenty}
	\twentyitem{2002-2006}{Southwest University of Political Science and Law}{Bachelor Of Lows}{}
\end{twenty}

\end{CJK}
\end{document} 
